\documentclass[12pt, a4paper]{article}
\usepackage[utf8]{inputenc}
\usepackage[spanish, es-tabla]{babel}
\usepackage{geometry}
\usepackage{graphicx}
\usepackage{amsmath, amssymb}
\usepackage{hyperref}
\usepackage{booktabs} % Para tablas profesionales
\usepackage{algorithm}
\usepackage{algpseudocode}
\usepackage{float}
\usepackage{caption}
\usepackage{xcolor}

% Configuración de hipervínculos
\hypersetup{
    colorlinks=true,
    linkcolor=blue,
    filecolor=magenta,
    urlcolor=cyan,
    citecolor=red,
}

% Configuración de márgenes
\geometry{top=2.5cm, bottom=2.5cm, left=3cm, right=3cm}

% Traducción de comandos de algoritmo al español
\floatname{algorithm}{Algoritmo}
\renewcommand{\algorithmicrequire}{\textbf{Entrada:}}
\renewcommand{\algorithmicensure}{\textbf{Salida:}}
\algrenewcommand\algorithmicwhile{\textbf{Mientras}}
\algrenewcommand\algorithmicfor{\textbf{Para}}
\algrenewcommand\algorithmicif{\textbf{Si}}
\algrenewcommand\algorithmicelse{\textbf{Sino}}
\algrenewcommand\algorithmicreturn{\textbf{Retornar}}

% Datos del documento
\title{
    \vspace{1cm}
    \Huge \textbf{Informe Técnico} \\
    \vspace{0.5cm}
    \Large Análisis de Complejidad y Diseño de Algoritmos para el Problema del Transporte Discreto \\
    \vspace{1.5cm}
    \large \textbf{Discrete Logistics} \\
    \vspace{2cm}
}
\author{
    \textbf{Autores:} \\
    Adrian Alejandro Souto Morales \\
    Gabriel Herrera Carrazana \\
}
\date{2025}

\begin{document}

    \begin{titlepage}
        \maketitle
        \thispagestyle{empty}
        \vfill
        \centering
        \textit{Proyecto Final de Diseño y Análisis de Algoritmos}
    \end{titlepage}

    \section{Introducción}
    En \textit{Discrete Logistics}, operamos en un nicho de mercado extremadamente delicado: el transporte transfronterizo de productos de valor incalculable. Este informe aborda el problema computacional de distribuir un conjunto de artículos de alto valor entre un número determinado de transportistas (``mulas''), respetando estrictamente la capacidad de carga y minimizando el riesgo financiero global.

    El objetivo es lograr una distribución del valor total de los artículos lo más equitativa posible, reduciendo el impacto de una pérdida catastrófica si un único transportista es comprometido.

    \section{Definición Formal del Problema}
    Dado un conjunto de $N$ artículos $A = \{a_1, ..., a_n\}$ y $M$ mulas $M = \{m_1, ..., m_m\}$, buscamos una partición tal que:

    \begin{enumerate}
        \item \textbf{Restricción de Capacidad:} $\sum_{a_i \in A_j} p_i \le C_j$ para toda mula $m_j$.
        \item \textbf{Función Objetivo (Equidad):} Minimizar $K$ tal que $|V_j - V_k| \le K$ para todo par de mulas $(j, k)$, donde $V_j$ es el valor total asignado a la mula $j$.
    \end{enumerate}

    \section{Análisis de Complejidad (Fase 2)}
    Para determinar la viabilidad computacional, analizamos la complejidad teórica del problema.

    \subsection{NP-Completitud}
    Demostramos que el problema pertenece a la clase \textbf{NP} y es \textbf{NP-Duro} mediante una reducción desde el \textbf{Problema de la Partición (PARTITION)}.

    \begin{itemize}
        \item \textbf{Construcción:} Dada una instancia de PARTITION (multiconjunto $S$), construimos una instancia de transporte con $M=2$ mulas, capacidad $T/2$ (donde $T = \sum S_i$) y umbral $K=0$.
        \item \textbf{Equivalencia:} Existe una solución al problema de transporte con diferencia 0 si y solo si existe una partición perfecta del conjunto $S$.
    \end{itemize}

    \textbf{Conclusión:} El problema es \textbf{NP-Completo}, lo que justifica el uso de heurísticas para instancias grandes.

    \section{Diseño de Algoritmos (Fase 3)}

    \subsection{Enfoque Exacto: Fuerza Bruta}
    Este algoritmo garantiza encontrar la solución óptima explorando todo el espacio de búsqueda.

    \begin{algorithm}[H]
        \caption{Fuerza Bruta para Transporte Discreto}
        \begin{algorithmic}[1]
            \Require Artículos $A$, Mulas $M$
            \Ensure Mejor Asignación $S_{opt}$
            \State $MejorDiferencia \gets \infty$
            \For{cada asignación $S$ posible de $A$ en $M$} \Comment{Complejidad $O(M^N)$}
            \If{$S$ cumple restricciones de peso}
                \State $Dif \gets \max(V(m)) - \min(V(m))$
                \If{$Dif < MejorDiferencia$}
                    \State $MejorDiferencia \gets Dif$
                    \State $S_{opt} \gets S$
                \EndIf
            \EndIf
            \EndFor
            \State \Return $S_{opt}$
        \end{algorithmic}
    \end{algorithm}
    \textbf{Complejidad:} $O(M^N \cdot (N + M^2))$. Inviable para $N > 12$.

    \subsection{Heurística Voraz (Greedy)}
    Estrategia eficiente que ordena los artículos por valor y los asigna a la mula con menor carga actual para balancear.

    \begin{algorithm}[H]
        \caption{Heurística Voraz (Greedy)}
        \begin{algorithmic}[1]
            \Require Artículos $A$, Mulas $M$
            \State Ordenar $A$ descendentemente por valor
            \For{cada artículo $art$ en $A$}
                \State $Candidata \gets$ Mula válida con menor $Valor_{actual}$
                \If{$Candidata$ existe}
                    \State Asignar $art$ a $Candidata$
                \Else
                    \State \Return Error (No hay solución factible)
                \EndIf
            \EndFor
            \State \Return Asignación final
        \end{algorithmic}
    \end{algorithm}
    \textbf{Complejidad:} $O(N \log N + N \cdot M)$.

    \subsection{Metaheurística: Búsqueda Local}
    Mejora la solución voraz aplicando movimientos de \textit{Relocate} y \textit{Swap} entre la mula más cargada y la menos cargada.

    \begin{algorithm}[H]
        \caption{Búsqueda Local (Hill Climbing)}
        \begin{algorithmic}[1]
            \Require Solución Inicial $S$
            \Repeat
                \State $Mejora \gets Falso$
                \State $M_{max} \gets$ Mula con mayor valor
                \State $M_{min} \gets$ Mula con menor valor
                \State \textbf{Intento 1:} Mover artículo de $M_{max}$ a $M_{min}$
                \If{movimiento reduce diferencia} $Mejora \gets Verdadero$ \EndIf
                \State \textbf{Intento 2:} Intercambiar artículos entre $M_{max}$ y $M_{min}$
                \If{intercambio reduce diferencia} $Mejora \gets Verdadero$ \EndIf
            \Until{$!Mejora$ o límite iteraciones}
        \end{algorithmic}
    \end{algorithm}

    \section{Análisis Experimental (Fase 4)}
    Los experimentos fueron realizados en Python comparando los tres enfoques.

    \subsection{Calidad de la Solución}
    Se probó con instancias pequeñas ($N=4$ a $10$, $M=2$) para comparar contra el óptimo global.

    \begin{table}[H]
        \centering
        \caption{Diferencia promedio respecto al Óptimo (Gap en \$)}
        \label{tab:resultados}
        \begin{tabular}{@{}ccc@{}}
            \toprule
            \textbf{N (Artículos)} & \textbf{Gap Greedy} & \textbf{Gap Búsqueda Local} \\ \midrule
            4  & 0.0  & 0.0 \\
            5  & 9.6  & 0.0 \\
            6  & 0.8  & 0.8 \\
            7  & 7.2  & 0.0 \\
            8  & 48.0 & 4.8 \\
            9  & 12.0 & 2.8 \\
            10 & 15.2 & 8.4 \\ \bottomrule
        \end{tabular}
    \end{table}

    \begin{figure}[H]
        \centering
        \includegraphics[width=0.85\textwidth]{calidad_solucion.png}
        \caption{El algoritmo Greedy presenta picos de error altos (ej. N=8), mientras que la Búsqueda Local reduce drásticamente esta brecha, acercándose al óptimo.}
    \end{figure}

    \subsection{Escalabilidad}
    Se realizaron pruebas de estrés con $N$ hasta 500.

    \begin{figure}[H]
        \centering
        \includegraphics[width=0.85\textwidth]{escalabilidad_tiempo.png}
        \caption{La Fuerza Bruta crece exponencialmente, volviéndose inútil para $N>15$. Los algoritmos aproximados se mantienen por debajo de 1 segundo incluso con 500 artículos.}
    \end{figure}

    \section{Conclusión}
    El análisis teórico y experimental confirma que el problema es intratable por métodos exactos para tamaños realistas. Sin embargo, la implementación de la **Metaheurística de Búsqueda Local** ha demostrado ser capaz de encontrar soluciones con un error promedio muy bajo ($< 5\%$ respecto al valor total en las pruebas) en tiempos de ejecución insignificantes, cumpliendo así con los requisitos operativos de \textit{Discrete Logistics}.

\end{document}